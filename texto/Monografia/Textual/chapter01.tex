\chapter{Introdução}

\section{Tema}

Este projeto trata do desenvolvimento de uma solução para o monitoramento de recursos e infraestrutura de redes de computadores.
A abordagem, caracterizada por sua compacidade e portabilidade, busca viabilizar a coleta, monitoramento e visualização de métricas em tempo real, facilitando a observabilidade de desempenho e utilização de elementos na rede .
% A partir de um único dispositivo de hardware, compacto e portátil, realiza-se a coleta, monitoramento e visualização de dados em tempo real, de forma a facilitar a observabilidade de elementos na rede e suas respectivas métricas.

\section{Delimitação}

A proposta aqui apresentada visa atender às demandas de usuários que desejam monitorar suas redes de computadores e os respectivos dispositivos conectados à esta. Essas necessidades podem surgir em diferentes contextos, como a administração contínua da infraestrutura de TI em empresas ou o acompanhamento pontual do funcionamento da rede em residências.
% Essa solução é voltada tanto para administradores de sistemas quanto para entusiastas de \textit{DevOps} em ambientes domésticos.

Uma boa ilustração da abrangência dessa solução é sua capacidade de operar tanto com equipamentos de rede corporativos (servidores, \textit{switches}, roteadores, \textit{access points}, \textit{storages}) quanto com dispositivos domésticos (\textit{notebooks}, computadores, \textit{smartphones}, dispositivos \textit{IoT}).


\section{Justificativa}
\section{Objetivos}
\section{Metodologia}
\section{Descrição}

Aenean purus arcu, auctor sed interdum vel, feugiat a lacus. Fusce nec nibh quis ipsum maximus finibus in at nulla. In non ultricies felis, ac interdum mi. Fusce eu congue sem. Nam congue aliquam libero, nec bibendum mi sollicitudin non. Suspendisse cursus ligula vitae nunc finibus, sed eleifend enim vestibulum. Proin facilisis leo rhoncus nisl hendrerit interdum. Aenean est ligula, gravida eget nunc et, vehicula pharetra dui. In eu massa egestas, auctor ante non, consequat odio. (\cite{alvim2007}).

Curabitur rhoncus blandit ipsum, id consequat urna venenatis id. Aliquam ex nisi, vestibulum quis tellus quis, ultricies egestas orci. Donec eu libero dui. Integer elementum felis et ligula congue rutrum. Sed a feugiat purus, eu rutrum massa. Vestibulum commodo elit id ornare pharetra. Pellentesque at pretium diam. Morbi sit amet placerat justo. Cras id nulla eros. Donec iaculis ligula eu gravida porttitor. Vestibulum tristique dapibus arcu porta euismod. Ut bibendum at nunc et interdum. In egestas pretium lacus, ut aliquet ipsum ultrices quis. Morbi euismod justo arcu, consequat sagittis orci aliquam in. Vivamus dignissim, libero vel accumsan viverra, odio erat venenatis sapien, a gravida quam neque vitae nisi.(\cite{mme2020}). 

\begin{figure}[H]
    \centering
    \includegraphics[width=0.5\linewidth]{Imagens/chap01/loren-ipsum-cover.jpg}
    \caption{Maecenas viverra convallis sem, id imperdiet neque rhoncus et. Ut vel mi erat. Nam quam arcu, mollis sodales felis at, sagittis iaculis lectus.}
    \label{fig:lorem_ipsum}
\end{figure}

\section{Organização do TCC}

In ornare, enim non porta interdum no Capítulo \ref{chap2},est lorem volutpat metus, pellentesque pharetra lacus est sed lacus. Vivamus quis magna et justo mattis commodo viverra in tellus. (Apêndice \ref{apendice}).

Aliquam convallis mauris sit amet elementum condimentum. Vestibulum eget tellus massa. Aenean nisl tortor, consequat ac lacus maximus, hendrerit consequat purus. Fusce aliquam, leo vel dictum molestie, lorem nibh aliquam diam, sit amet accumsan justo ante sit amet tellus no Capítulo \ref{chap6}.
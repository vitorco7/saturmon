\chapter{Introdução}

\section{Tema}

Este projeto propõe o desenvolvimento de uma solução voltada ao monitoramento da saturação de recursos em dispositivos conectados a uma rede de computadores. Fundamentado exclusivamente em ferramentas de código aberto e na adoção do paradigma de Infraestrutura como Código (IaC), privilegia-se escalabilidade e replicabilidade nos processos de coleta, monitoramento e visualização de métricas. Assim, facilita-se a observabilidade da utilização do hardware dos dispositivos monitorados, com objetivo de auxiliar na administração e manutenção da infraestrutura da rede, seja ela corporativa ou doméstica.

\section{Delimitação}

A solução destaca-se em versatilidade, podendo ser empregada em diferentes cenários. Ela contempla desde o monitoramento de dispositivos em infraestruturas de TI tradicionais — como servidores, roteadores, switches e storages — até o acompanhamento de equipamentos em ambientes domésticos, incluindo computadores pessoais, smartphones, dispositivos de Internet das Coisas (IoT) ou até mesmo dispositivos Edge.

No entanto, diante da indisponibilidade de equipamentos físicos durante o desenvolvimento do projeto, adotou-se um escopo mais restrito. Para isso, implementou-se virtualmente uma rede doméstica composta por cinco desktops, utilizando contêineres Docker com diferentes especificações de CPU, memória e sistema operacional. Paralelamente, optou-se por uma estratégia de isolamento mais rigorosa dos recursos dos dispositivos virtuais, visando aproximar o comportamento desses ambientes simulados de um dispositivo físico real.

Além dos dispositivos simulados, incluiu-se também um computador físico, com o objetivo de enriquecer a análise e proporcionar uma base qualitativa para comparação das métricas obtidas nos dispositivos virtuais.

Como consequência das limitações impostas, inviabilizou-se a coleta de determinadas métricas dos dispositivos virtuais, especialmente aquelas relacionadas ao armazenamento, como espaço disponível e operações de entrada e saída (I/O) em disco.

\section{Justificativa}

A relevância deste projeto reside na sua capacidade em atender uma crescente demanda por soluções de monitoramento eficazes, acessíveis, replicáveis e escaláveis de dispositivos conectados em redes heterogêneas. Ao empregar exclusivamente ferramentas de código aberto, a proposta democratiza o acesso a práticas avançadas de monitoramento, eliminando restrições impostas por soluções proprietárias e promovendo a adoção de padrões abertos e interoperáveis.

Sob a ótica da Engenharia de Confiabilidade de Sites (SRE), destaca-se o conceito de saturação, que se refere à aproximação dos limites de capacidade de um recurso, como CPU, memória, armazenamento ou largura de banda.  A detecção proativa da saturação é fundamental para evitar falhas, degradações de desempenho e impactos negativos na experiência do usuário. O projeto oferece uma estrutura robusta para a identificação dessas condições, por meio da coleta contínua e sistemática de métricas relevantes, possibilitando a implementação de ações preventivas e corretivas antes que o sistema atinja um estado crítico.

A adoção do paradigma de Infraestrutura como Código (IaC) constitui outro pilar central da proposta. Ao automatizar a definição, o provisionamento e o gerenciamento da infraestrutura de monitoramento por meio de código, o projeto assegura reprodutibilidade, versionamento e portabilidade, além de minimizar erros humanos e aumentar a eficiência operacional. Essa abordagem não só facilita a implantação da solução em múltiplos ambientes — sejam eles físicos, virtuais ou em nuvem —, como também favorece a manutenção e a evolução contínua da infraestrutura monitorada, alinhando-se às melhores práticas contemporâneas de gestão de TI.

Por fim, o foco em observabilidade amplia a capacidade de compreensão do comportamento dos sistemas monitorados. Diferentemente do simples monitoramento, a observabilidade oferece uma visão holística e integrada dos dados coletados, permitindo a identificação proativa de anomalias, gargalos e tendências de saturação. Isso subsidia decisões informadas, baseadas em dados, e contribui para a otimização contínua do desempenho e da confiabilidade da infraestrutura de rede.

Em síntese, ao integrar os princípios de saturação de SRE, Infraestrutura como Código e observabilidade, este projeto não apenas supre uma lacuna técnica relevante, mas também promove a disseminação de práticas modernas e eficientes de gestão de redes, em consonância com as demandas atuais por transparência, automação e sustentabilidade operacional.

\section{Objetivos}
\section{Metodologia}
\section{Descrição}
\chapter{Metodologia Proposta}
\label{chap3}

{\color{red}
No capítulo anterior, foram apresentados os conceitos e as tecnologias relevantes para a compreensão da solução de monitoramento desenvolvida. Contudo, vale ressaltar que nem todas as ferramentas descritas anteriormente foram efetivamente empregadas na implementação final da proposta.

Este capítulo tem por objetivo detalhar todo o processo de desenvolvimento da solução de monitoramento, desde as primeiras considerações e escolhas de arquitetura, passando pela seleção e adaptação das ferramentas adotadas, até a apresentação da solução final implementada. Serão discutidas as motivações para determinadas decisões técnicas, desafios encontrados ao longo do desenvolvimento, e os métodos empregados para contorná-los, buscando sempre respaldar as opções feitas com base nos conceitos apresentados no Capítulo 2.

Também são descritos os equipamentos utilizados durante o desenvolvimento. Foram utilizados dois computadores, cujas especificações são apresentadas na Tabela \ref{tab:available-hardware}.


\begin{table}[H]
\centering
{\color{red}
\caption{Especificações de hardware dos equipamentos disponíveis}
\label{tab:available-hardware}
\begin{tabular}{lcc}
\toprule
\textbf{Componente} & \textbf{Desktop} & \textbf{Notebook} \\
\midrule
CPU   & Intel Core i5-7600   & Intel Core i7-8565U \\
RAM   & 16GB                 & 32GB                \\
Disco & 500GB (SSD)            & 250GB (SSD)          \\
\bottomrule
\end{tabular}
}
\end{table}

As especificações do sistema operacional serão apresentadas posteriormente, visto que a escolha desse aspecto evoluiu ao longo do desenvolvimento, conforme será detalhado neste capítulo.

\section{Abordagem Preliminar}
\label{section:AbordagemPreliminar}

Esta seção apresenta uma visão geral das decisões iniciais tomadas antes da definição da arquitetura final da solução de monitoramento. As primeiras etapas deste trabalho podem ser divididas em dois momentos: inicialmente, o escopo era voltado para o monitoramento de infraestruturas de TI tradicionais (como servidores e clusters); em seguida, evoluiu para englobar também dispositivos conectados à rede de forma abrangente, sem restrições de tipagem.

\subsection{Escopo inicial - Infraestrutura Tradicional}
\label{subsection:EscopoInicial}

Durante as discussões preliminares, o objetivo era monitorar servidores. Foi considerada uma arquitetura onde um equipamento -- podendo ser um NUC ou Raspberry Pi -- realizaria a coleta de métricas deste servidor e, após o processamento destes dados, disponibilizaria-os em painés com visualizações gráficas especializadas. Alternativas com Orange Pi como equipamento foram analisadas, mas a escolha final recaiu sobre o NUC ou Raspberry Pi devido à sua popularidade e suporte.

Entretanto, como não foi possível adquirir fisicamente um NUC ou Raspberry Pi, decidiu-se simular toda a \foreign{stack} de monitoramento em uma máquina virtual.

No desktop disponível, à época com sistema operacional não-Linux, instalou-se o VMWare Workstation Player para configuração de uma VM com Rocky Linux. A escolha pelo Rocky Linux baseou-se em sua compatibilidade com o CentOS, estabilidade e longo ciclo de suporte, características valorizadas em ambientes corporativos.

No entanto, o VMWare Workstation Player mostrou-se limitado em recursos, dificultando a execução de múltiplas VMs. Por isso, migrou-se para o VirtualBox, o que trouxe maior flexibilidade e controle.

Esta estratégia inicial apresentou limitações importantes: a primeira era a maior dificildade na aquisição de dados, pois seria necessário obter um servidor físico para coleta dos mesmos. Outra limitação seria elevado custo computacional inerente ao uso de VMs em comparação a soluções baseadas em contêineres, e pouca modularidade, resultando em manutenção complexa e baixa praticidade

Devido a esses fatores, verificou-se a necessidade de reavaliar o escopo e buscar abordagens mais aderentes aos recursos disponíveis e aos objetivos do projeto.


\begin{figure}[H]
\centering
\fbox{\rule{0pt}{150pt} \rule{200pt}{0pt}} % 200pt wide, 150pt high
\caption{Image placeholder}
\label{fig:placeholder}
\end{figure}

-----
****Inserir aqui uma imagem esquemática da arquitetura inicial****
-----

\subsection{Escopo Reformulado - Monitoramento Amplo}
\label{subsection:EscopoReformulado}




}
\section{Descrição Macro da Arquitetura da Solução}

Placeholder

\section{Infraestrutura}

Placeholder

\section{Discussão sobre as métricas}

Placeholder

\section{Aplicação de Monitoramento}

Placeholder
